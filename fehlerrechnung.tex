\documentclass[titlepage]{scrartcl}
% Code Darstellung
\usepackage{listings}
\usepackage{listingsutf8}
\usepackage{multicol}

%lange Tabellen
\usepackage{longtable}
%Referenzen zwischen unterschiedlichen Dateien
\usepackage{xr}
\externaldocument{theorie}
\usepackage{lscape}
%Deutsche Sprachunterstützung
\usepackage[utf8]{inputenc}
\usepackage[ngerman]{babel}
\usepackage{marvosym}
\DeclareUnicodeCharacter{20AC}{\EUR}

%Für das Einbinden von Bildern
\usepackage{graphicx}

%Tabellen
\usepackage{array}

%Tabellen automatisch schoener
\usepackage{booktabs}

%Caption
\usepackage{caption}
\usepackage{subcaption}

%Formeln
\usepackage{mathtools}
\usepackage{amsmath}
\usepackage{amssymb}
\usepackage{amstext}
\usepackage{dsfont}

%\usepackage{mnsymbol}

% Interssante natbib Optionen: 
% numbers : Nummerierte Zitateinheiten
% sort&compress : Bei mehrfachen Zitaten, Sortierung und ggf. Verkürzungen
%\usepackage[]{natbib}

%Vectorpfeile schöner
\usepackage{esvect}

%Formatierung
\usepackage[T1]{fontenc}
\usepackage{lmodern}
\usepackage{microtype}

%Schaltbilder malen
%\usepackage[europeanresistors,cuteinductors,siunitx]{circuitikz}
\usepackage{comment}
\usepackage{csquotes}

%Formatierungsanweisungen
\newcommand{\wichtig}[1]{\underline{\large{#1}}}
\newcommand{\aref}[1]{Abb. \ref{#1}}
\newcommand{\R}{\mathbb{R}}
\newcommand{\K}{\mathbb{K}}
\newcommand{\C}{\mathbb{C}}

%Klickbare Referenzen
%\usepackage[hidelinks]{hyperref}

\begin{document}

\section{Aufgabe 1}
\subsection{Teilaufgabe a)}

Die Formel zur Berechnung von $ d $ ist $ d = 1/ \pi $. Daraus ergibt sich $ \frac{\partial d}{\partial \pi} = - \frac{1}{\pi^2} $. Laut Fehlerfortpflanzung gilt:

\begin{equation}
\delta d = \sqrt{\frac{1}{\pi^4 \cdot \delta \pi^2}} = \vert \frac{\delta \pi}{\pi^2} \vert
\label{form:fehlerfort1}
\end{equation}

Damit ergibt sich

\begin{equation}
d = \frac{1}{0.3107} pc = 3.219 pc
\end{equation}

und

\begin{equation}
\delta d = \vert \frac{0.0009}{0.3107^2} \vert pc = 0.010 pc
\end{equation}

Die Entfernung des Sterns ist also:

\begin{equation}
d = (3.219 \pm 0.010 ) pc
\end{equation}

\subsection{Teilaufgabe b)}
Aufgrund des relativen Fehlers ergibt sich für $ \pi = (1.0 \pm 0.6) mas $. Damit ergibt sich

\begin{equation}
d = \frac{1}{0.001} pc = 1000 pc
\end{equation}

und mit \eqref{form:fehlerfort1}

\begin{equation}
\delta d = \vert \frac{0.0006}{0.001^2} \vert pc = 600 pc
\end{equation}

Die Enfernung von Deneb ist damit:

\begin{equation}
d = (1.0 \pm 0.6 ) kpc
\end{equation}

\subsection{Teilaufgabe c)}
\subsubsection{Teilaufgabe (i)}
%$ \log_{10} (5) $
Die Beziehung $ (m - M)_V - A_V = 5 \cdot \log_{10} (d) - 5 $ lässt sich umformen zu:

\begin{equation}
d = 10 \cdot {10}^{\frac{m_V - M_V - A_V}{5}}
\end{equation}

Für die Berechnung des Fehlers wird berechnet:

\begin{equation}
\frac{\partial d}{\partial M_V} = -2 \cdot \ln (10) \cdot {10}^{\frac{m_V - M_V - A_V}{5}} = - \frac{\ln (10)}{5} \cdot d
\label{form:fehlerfort2}
\end{equation}

Mit \eqref{form:fehlerfort2} ergibt sich dann für die Fehlerfortpflanzung:

\begin{equation}
\delta d = \sqrt{(- \frac{\ln (10)}{5} \cdot d)^2 \cdot {\delta M_V}^2} = \frac{\ln (10)}{5} \cdot \vert d \cdot \delta M_V \vert
\end{equation}

So ergibt sich die Entfernung von Deneb zu:

\begin{equation}
d = 10 \cdot {10}^{\frac{1.25 - (-8.27) - 0.113}{5}} = 761 pc
\end{equation}

Für den Fehler erhält man:

\begin{equation}
\delta d = \frac{\ln (10)}{5} \cdot \vert 761 \cdot 0.23 \vert pc = 90 pc
\end{equation}

Der Abstand von Deneb ergibt sich also mit dieser Methode zu:

\begin{equation}
d = (761 \pm 90) pc
\end{equation}

\subsubsection{Teilaufgabe (ii)}
Das Stefan-Boltzmann-Gesetz lässt sich umformen zu:

\begin{equation}
R = \sqrt{\frac{L}{4 \pi \sigma}} \cdot \frac{1}{T_{eff}^2}
\label{form:boltzmann}
\end{equation}

Um \eqref{form:boltzmann} in den Einheiten $ L_{\odot} $ und $ T^2_{eff,\odot} $ ausdrücken, wird \eqref{form:boltzmann} so erweitert, dass sich ergibt:

\begin{equation}
R = \sqrt{\frac{L/L_{\odot}}{4 \pi \sigma}} \cdot \frac{1}{T^2_{eff}/T^2_{eff,\odot}} \cdot \frac{\sqrt{L_{\odot}}}{T^2_{eff,\odot}}
\end{equation}

Definiert man $ L[L_{\odot}] := L/L_{\odot} $, $ T_{eff}[T_{eff,\odot}] := T_{eff}/T_{eff,\odot} $ und $ R[R_{\odot}] := R/R_{\odot} $ als Leuchtkraft, effektive Temperatur und Radius in Sonneneinheiten, so ergibt sich:

\begin{equation}
R[R_{\odot}] = \frac{\sqrt{L[L_{\odot}]}}{T^2_{eff}[T_{eff,\odot}]} \cdot \frac{\sqrt{L_{\odot}}}{\sqrt{4 \pi \sigma} \cdot T^2_{eff,\odot} \cdot R_{\odot}}
\label{form:stefanUmgestellt}
\end{equation}

Mit dem Stefan-Boltzmann-Gesetz (Gl. \eqref{form:boltzmann}) folgt, dass:

\begin{equation}
\frac{\sqrt{L_{\odot}}}{\sqrt{4 \pi \sigma} \cdot T^2_{eff,\odot} \cdot R_{\odot}} = 1
\end{equation}

Man erhält dann aus \eqref{form:stefanUmgestellt} die Formel des Radius in Abhängigkeit der Leuchtkraft und der Effektivtemperatur in Einheiten der Sonne:

\begin{equation}
R[R_{\odot}] = \frac{\sqrt{L[L_{\odot}]}}{T^2_{eff}[T_{eff,\odot}]}
\end{equation}

\newpage

Für die Fehlerfortpflanzung berechnet man

\begin{equation}
\frac{\partial R}{\partial T_{eff}} = - \frac{2 \sqrt{L}}{T^3_{eff}}
\label{form:fehlerfortStefan1}
\end{equation}

und

\begin{equation}
\frac{\partial R}{\partial L} = \frac{1}{2 \sqrt{L} \cdot T^2_{eff}}
\label{form:fehlerfortStefan2}
\end{equation}

Aus \eqref{form:fehlerfortStefan1} und \eqref{form:fehlerfortStefan2} folgt dann:

\begin{equation}
\delta R = \sqrt{(\frac{\partial R}{\partial T_{eff}})^2 \cdot \delta T^2_{eff} + (\frac{\partial R}{\partial L})^2 \cdot \delta L^2} = \sqrt{\frac{4 L}{T^6_{eff}} \cdot \delta T^2_{eff} + \frac{1}{4 L T^4_{eff}} \cdot \delta L^2}
\end{equation}

Für den Radius von Deneb ergibt sich dann aus den vorigen Überlegungen:

\begin{equation}
R[R_{\odot}] = \frac{\sqrt{1.8 \cdot 10^5}} {(\frac{8530}{5778})^2} = 195
\end{equation}

und

\begin{equation}
\delta R = \sqrt{\frac{4 \cdot 1.8 \cdot 10^5}{({\frac{8530}{5778}})^6} \cdot (\frac{80}{5778})^2 + \frac{1}{4 \cdot 1.8 \cdot 10^5 \cdot ({\frac{8530}{5778}})^4} \cdot (0.4 \cdot 10^5)^2} = 22
\end{equation}

%\delta R = \sqrt{13.33303+467.84449}

\newpage

\begin{thebibliography}{9}
\bibitem[Bec]{kmann} BECKMANN, Dieter. Astrophysik. C.C.Buchner, 2011.
\bibitem[Ort]{szeit} Wikipedia: Ortszeit. Online im Internet: URL: http://de.wikipedia.org/wiki/Ortszeit (Stand: 01.03.2014). 


\end{thebibliography}

\end{document}